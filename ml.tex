\chapter{Машинное обучение}
\label{ch:ml}

\section{Кластеризация}
\emph{Кластеризация} — метод обнаружения групп (кластеров) связанных между собой объектов. Кластеризация — пример обучения без учителя.

\subsection{Алгоритм $k$ средних}
Пусть имеется тестовый набор данных $x^{(1)}, \dots, x^{(m)}$, где $x^{(i)} \in \mathbb{R}^n$. Наша цель — выделить $k$ групп из этого набора, где $k$ — предварительно задано. Тогда алгоритм $k$ средних следующий:
\begin{enumerate}
  \item Выбрать случайно центроиды кластеров $\mu^{(1)}, \dots, \mu^{(k)}$, где $\mu^{(i)} \in \mathbb{R}^n$.
  \item Повторять до сходимости (до тех пор, пока центры кластеров не станут постоянными):
    \begin{enumerate}
      \item Для каждого $i$, присвоить \[ c^{(i)} = \operatorname*{arg\,min}_j \| x^{(i)} - \mu_j \|^2. \]
      \item Для каждого $j$, присвоить \[ \mu_j = \frac{\sum_{i = 1}^{m}{1\{ c^{(i)} = j \} x^{(i)}}}{\sum_{i = 1}^{m}{1\{ c^{(i)} = j \}}}. \]
    \end{enumerate}
\end{enumerate}

Этот же алгоритм на пальцах:
\begin{enumerate}
  \item Проинициализировать.
  \item Классифицировать точки по ближайшему к ним центру кластера.
  \item Перевычислить каждый из центров.
  \item Если центры изменились, то повторить с пункта $2$.
\end{enumerate}
