\chapter{Базовые структуры данных и алгоритмы}
\label{ch:basic-ds}

\section{$O$-нотация}
\label{sec:o-notation}

Размер входных данных. Время работы алгоритма. Время работы в наихудшем случае, в среднем.

\textbf{Порядок роста}, или \textbf{скорость роста}. Пусть время работы алгоритма в наихудшем случае выражается формулой $an^2 + bn + c$, где $a$, $b$ и $c$ --- некоторые константы. Поскольку при больших $n$ членами меньшего порядка можно пренебречь, то рассматривается только главный член формулы $n^2$. Таким образом, время работы алгоритма в наихудшем случае равно $\Theta(n^2)$.

\subsection{Обозначения}
\begin{tabular}{lp{8cm}}
  \toprule
  Обозначение & Объяснение \\
  \midrule
  $f(n) \in O(g(n))$ & $f$ ограничена сверху функцией $g$ (с точностью до постоянного множителя) асимптотически \\
  $f(n) \in \Omega(g(n))$ & $f$ ограничена снизу функцией $g$ (с точностью до постоянного множителя) асимптотически \\
  $f(n) \in \Theta(g(n))$ & $f$ ограничена снизу и сверху функцией $g$ асимптотически \\
  $f(n) \in o(g(n))$ & $g$ доминирует над $f$ асимптотически \\
  $f(n) \in \omega(g(n))$ & $f$ доминирует над $g$ асимптотически \\
  \bottomrule
\end{tabular}

\section{Двоичный поиск}
\label{sec:binary-search}
\section{Сортировка}
\label{sec:sorting}
\section{Расширяемые массивы}
\label{sec:ext-arrays}
\section{Списки}
\label{sec:lists}
\section{Деревья}
\label{sec:trees}
\section{Хэш-таблицы}
\label{sec:hash-tables}
\section{Динамическое программирование}
\label{sec:dyn-programming}
\section{Некоторые заметки о языке C}
\label{sec:c-notes}
