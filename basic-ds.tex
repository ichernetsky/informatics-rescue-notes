\chapter{Базовые структуры данных и алгоритмы}
\label{ch:basic-ds}

\section{$O$-нотация}
\label{sec:o-notation}

\textbf{Размер входных данных} зависит от рассматриваемой задачи.

\textbf{Время работы} алгоритма измеряется в количестве элементарных операций. Оно зависит от размера входных данных.

\textbf{Порядок роста}, или \textbf{скорость роста}. Пусть время работы алгоритма в наихудшем случае выражается формулой $an^2 + bn + c$, где $a$, $b$ и $c$ --- некоторые константы. Поскольку при больших $n$ членами меньшего порядка можно пренебречь, то рассматривается только главный член формулы $n^2$. Таким образом, время работы алгоритма в наихудшем случае равно $\Theta(n^2)$.

При \emph{асимптотическом анализе} нас интересует то, как растет время выволнения алгоритмы с увеличением размера входных данных \emph{в пределе}.

\subsection{Обозначения}
\begin{tabular}{lp{8cm}}
  \toprule
  Обозначение & Объяснение \\
  \midrule
  $f(n) \in O(g(n))$ & $f$ ограничена сверху функцией $g$ (с точностью до постоянного множителя) асимптотически \\
  $f(n) \in \Omega(g(n))$ & $f$ ограничена снизу функцией $g$ (с точностью до постоянного множителя) асимптотически \\
  $f(n) \in \Theta(g(n))$ & $f$ ограничена снизу и сверху функцией $g$ асимптотически \\
  $f(n) \in o(g(n))$ & $g$ доминирует над $f$ асимптотически \\
  $f(n) \in \omega(g(n))$ & $f$ доминирует над $g$ асимптотически \\
  \bottomrule
\end{tabular}

\subsection{Определения}
\begin{align}
  f(n) \in O(g(n)) \quad = \quad &\exists c > 0, n_0 \quad \forall n > n_0 \quad f(n) \leq c \cdot g(n),\\
  f(n) \in \Omega(g(n)) \quad = \quad &\exists c > 0, n_0 \quad \forall n > n_0 \quad c \cdot g(n) \leq f(n),\\
  f(n) \in \Theta(g(n)) \quad = \quad &\exists c_1 > 0, c_2 > 0, n_0 \quad \forall n > n_0 \nonumber\\
                                      &c_1 \cdot g(n) \leq f(n) \leq c_2 \cdot g(n),\\
  f(n) \in o(g(n)) \quad = \quad &\forall \varepsilon > 0 \quad \exists n_0 \quad \forall n > n_0 \quad f(n) < \varepsilon \cdot g(n),\\
  f(n) \in \omega(g(n)) \quad = \quad &\forall c > 0 \quad \exists n_0 \quad \forall n > n_0 \quad c \cdot g(n) < f(n).
\end{align}

\section{Двоичный поиск}
\label{sec:binary-search}
\section{Сортировка}
\label{sec:sorting}
\section{Расширяемые массивы}
\label{sec:ext-arrays}
\section{Списки}
\label{sec:lists}
\section{Деревья}
\label{sec:trees}
\section{Хэш-таблицы}
\label{sec:hash-tables}
\section{Динамическое программирование}
\label{sec:dyn-programming}
\section{Некоторые заметки о языке C}
\label{sec:c-notes}
