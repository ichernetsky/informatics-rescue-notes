\chapter{Параллельное программирование}
\label{ch:concurrent}



\section{Основные понятия параллельного программирования}

\subsection{Основные проблемы}

Управление одновременными действиями и их возможным взаимовлиянием приводит к проблемам четырех типов:
\begin{itemize}
\item \emph{Синхронизация} --- это процесс, при помощи которого два или более программных потока координируют свои действия. Например, один поток, чтобы продолжить выполнение, ждет,когда другой закончит свое задание.
\item \emph{Взаимодействие} --- обмен данными между программными потоками, с которым связаны проблемы ширины полосы пропускания и задержек.
\item \emph{Балансировка нагрузки} --- распределение работы между несколькими програмными потоками так, чтобы все они выполняли примерно одинаковый объем работы.
\item \emph{Масштабируемость} --- проблема эффективного использования большего числа программных потоков при запуске программы на более мощной системе. Например, если программа написана для эффективного использования четырех ядер, будет ли она эффективно работать на системе с восемью процессорными ядрами.
\end{itemize}



\section{Примитивы синхронизации}



\section{POSIX API}



\section{Управление потоками}



\section{Проблемы и их решение}



\section{Другие подходы: передача сообщений и транзакционная память}
